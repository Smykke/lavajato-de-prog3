\documentclass[12pt,a4paper]{article}
\usepackage[brazil]{babel}
\usepackage[T1]{fontenc}
\usepackage[utf8]{inputenc}
\usepackage{amsfonts}
\usepackage{amssymb}
\usepackage{hyperref}
\hypersetup{
    colorlinks=true,
    linkcolor=blue,
    urlcolor=cyan,
}
\urlstyle{same}

\author{Gabriel Ferrari e Marcela Vieira}
\title{vereadores-GabrielFerrari-MarcelaVieira}

\begin{document}
\begin{center}\textbf{
\large UFES - CENTRO TECNOLÓGICO\\ DEPARTAMENTO DE INFORMÁTICA \\
\normalsize Programação III
\vfill
GABRIEL FERRARI\\
MARCELA VIEIRA}
\vfill
\begin{center}
Processamento de dados da Eleição 2016 de vereadores
\end{center}
\vfill
\textbf{
\large Vitória\\
11 de novembro de 2016}
\pagestyle{empty}
\end{center}
\newpage

\section{Introdução}
O sistema  eleitoral brasileiro utiliza dois métodos de eleição de candidatos para posições de liderança no País: majoritária e proporcional.\\
O programa criado analisa os resultados das eleições exportados pelo \textit{software} Divulga 2016, disponibilizado pela Justiça Eleitoral, e apresenta relaórios de diferentes resultados
alternando-se os dois métodos de escolha de candidatos e comparando-os.\\
Todos os dados provém do arquivo \texttt{CSV} exportado e são processados pelo programa escrito em linguagem \texttt{Java}.

\section{Objetivos}
Gerar relatórios na saída padrão a partir da leitura e manipulação de um arquivo de divulgação, determinado na linha de comando, acerca dos resultados de diversos Estados das Eleições 2016 de vereadores.\\

\section{Metodologia}
O programa lê o arquivo de entrada, cria listas de vereadores, cada qual com informações de nome, posição na divulgação final, partido, número de votos, entre outras, e listas de partidos e coligações.\\
Essas listas são ordenadas por número de votos e o programa gera novos \textit{rankings} para as Eleições de acordo com os diferentes métodos de votação. É possível saber os diferentes resultados para os candidatos, partidos e coligações.

\subsection{Candidato.java}
Abstração dos candidatos à vereador cujos atributos são retirados da leitura individual das linhas do documento de entrada, exceto por \texttt{eleito}, que é um atributo booleano que facilita a identificação de candidatos eleitos em outras métodos do programa.\\
\texttt{partido} e \texttt{coligação} referem-se a instâncias específicas de Partido e Coligação. Esse \textit{link} permite maior coesão entre as classes e aplicam-se bem ao contexto, pois todos os candidatos pertencem a um partido, os quais se relacionam dentro de coligações.\\

\subsection{Partido.java}
Abstração dos partidos contidos nas informações de cada candidato. Um partido tem nome, a quantidade total de votos recebidos pelos candidatos que o compõem e a lista de candidatos pertencentes a ele.\\

\subsection{Coligacao.java}
Abstração de coligações (conjunto de partidos), também referenciadas nas informações de entrada dos candidatos. Uma coligação tem nome, quantidade total de votos recebidos pelos candidatos dos partidos que a formam e a lista de partidos da coligação.\\

\subsection{Eleicao.java}
Esta classe contém as listas de partidos, vereadores e coligações de uma eleição e métodos para criar os diferentes relatórios descritos na especificação do trabalho.\\
As primeiras funções são de criação, adição de objetos e outras comuns ao uso de listas. Na seção seguinte do arquivo, temos métodos específicos para cada relatório, que retornam os resultados para os métodos de impressão de \texttt{Leitor.java}.

\subsection{Leitor.java}
É a classe principal do programa, com as chamadas de leitura, criação e escrita dos resultados. Os métodos foram separados entre \texttt{Eleicao} e \texttt{Leitor} para que a classe principal contesse apenas a \texttt{main} e métodos estritamente necessários, garantindo a organização e clareza no entendimento do programa.

\section{Resultados e Avaliação}
O programa, como mencionado, recebe um arquivo \texttt{CSV}, faz a leitura e, para cada linha, cria um candidato com as informações contidas nessa linha e o adiciona à lista de candidatos. O método de leitura do candidato se encarrega de adicionar os partidos e coligações às suas respectivas listas.\\
Os relatórios gerados pelo programa são:\\

\subsubsection*{Número de Vagas}
A contagem do número de vagas para vereador de uma cidade é realizada pela contagem de candidatos que apresentam um asterisco (*) antes do seu respectivo índice no arquivo de entrada. Os asteriscos representam os candidatos eleitos, logo, a quantidade de eleitos é igual a de vagas.\\
Durante a leitura de candidatos, o objeto recebe o valor \texttt{true} no atributo \texttt{eleito} caso apresente o asterisco. A função \texttt{getNumVagas} passa por toda a lista de candidatos contando os candidatos classificados como "eleito".

\subsubsection*{Candidatos Eleitos}
A função \texttt{getEleitos} cria uma lista com todos os candidatos eleitos (segundo o mesmo método citado anteriormente) e a retorna para impressão. Os candidatos da lista são escritos conforme a formatação abaixo:
\begin{center}
\texttt{1 - FABRÍCIO GANDINI (PPS , 7611 votos)	- Coligação:  PPS / PROS}\\
\end{center}

\subsubsection*{Candidatos mais votados dentro do número de vagas}
Para apresentar os candidatos mais votados, é necessário apenas ordená-los por número decrescente de vagas. Para reordená-los com esse critério sem alterar a lista original, a função \texttt{getMaisVotados} cria uma \textit{cópia} da lista original com o método \texttt{clone()} e reordena a lista copiada com o comparador por número de votos sobrescrito em \texttt{Candidato.java}.\\
O índice de cada candidato também é alterado segundo a nova ordem da lista.

\subsubsection*{Candidatos não eleitos e que seriam eleitos se a votação fosse majoritária}
Este relatório é gerado pela função \texttt{getPrejudicados}, que clona a lista de candidatos e a ordena pelo número de votos (de forma decrescente). Desconsiderando os candidatos que estão nas N primeira posições (em que N é o número de vagas), já que esses foram eleitos, analisa os próximos candidatos da lista que tem valor \texttt{false} no atributo \texttt{eleito}.


\subsubsection*{Candidatos que não seriam eleitos se a eleição fosse majoritária}
Com a lista clonada ordenada por número decrescente de votos, a função \texttt{getBeneficiados} retorna os candidatos eleitos a partir da posição N+1 (em que N é a quantidade de vagas).

\subsubsection*{Total de votos por coligação/partido e número de candidatos eleitos}
Este relatório envolve, simplesmente, a impressão da lista de coligações ordenada por número decrescente de votos na sua criação. Dentro de \texttt{toString}, a função \texttt{getNEleitos} soma a quantidade de candidatos eleitos fazendo uma iteração sobre a lista de partidos que fazem parte da coligação.

\subsubsection*{Total de votos por partido e número de candidatos eleitos}
Funciona de forma análoga ao relatório anterior, mas utilizando a lista de partidos.

\subsubsection*{Total de votos nominais}
A função \texttt{totalVotosNominais} percorre a lista de partidos somando os valores no atributo "votos" de cada partido e retorna o total.\\

\\Os testes realizados com os arquivos disponibilizados para as três cidades mostram os resultados esperados. Os arquivos com mais detalhes se encontram no \href{https://github.com/Smykke/lavajato-de-prog3}{repositório relacionado}.\\
\textit{Known bugs:} há um erro de formatação na impressão do resultado dos partidos e coligações, que apresenta \textit{'s'} no fim de "candidato" e "eleito" nas quantidades 0 e 1  apesar de haver um comparador lógico para essas situações.\\

\end{document}
